\documentclass[10pt]{beamer}
\usetheme[
%%% option passed to the outer theme
%    progressstyle=fixedCircCnt,   % fixedCircCnt, movingCircCnt (moving is deault)
  ]{Feather}

% If you want to change the colors of the various elements in the theme, edit and uncomment the following lines

% Change the bar colors:
%\setbeamercolor{Feather}{fg=red!20,bg=red}

% Change the color of the structural elements:
%\setbeamercolor{structure}{fg=red}

% Change the frame title text color:
%\setbeamercolor{frametitle}{fg=blue}

% Change the normal text color background:
%\setbeamercolor{normal text}{fg=black,bg=gray!10}

%-------------------------------------------------------
% INCLUDE PACKAGES
%-------------------------------------------------------

% General
\usepackage[utf8]{inputenc}
\usepackage[portuguese]{babel}
\usepackage[T1]{fontenc}
\usepackage{helvet}

% Code syntax highlight
\usepackage{setspace}
\usepackage{color}
\usepackage{listings}
\usepackage{lstlinebgrd}

% Specify table cell length
\usepackage{array}

% Hyperlink reference
\usepackage{hyperref}

%-------------------------------------------------------
% DEFFINING AND REDEFINING COMMANDS
%-------------------------------------------------------

% colored hyperlinks
\newcommand{\chref}[2]{
  \href{#1}{{\usebeamercolor[bg]{Feather}#2}}
}

%-------------------------------------------------------
% Configure syntax highlight
%-------------------------------------------------------
\definecolor{codepurple}{rgb}{0,0.6,0}
\lstset{
  backgroundcolor=\color{white},
  breaklines=true,
  commentstyle=\color{green},
  extendedchars=true,
  frame=single,
  keepspaces=true,
  keywordstyle=\color{magenta},
  language=Ruby,
  numbers=left,
  numbersep=10pt,
  numberstyle=\small\color{gray},
  rulecolor=\color{black},
  stringstyle=\color{blue},
  tabsize=2
}
%-------------------------------------------------------
% INFORMATION IN THE TITLE PAGE
%-------------------------------------------------------

% [] is optional - is placed on the bottom of the sidebar on every slide
% is placed on the title page
\title[] { 
\textbf{Kuniri Project}
}

\subtitle[An overview] {
}

\author[kuniri presentation] {
Rodrigo Siqueira Jordão
}

\date{\today}

%-------------------------------------------------------
% THE BODY OF THE PRESENTATION
%-------------------------------------------------------

\begin{document}

%-------------------------------------------------------
% THE TITLEPAGE
%-------------------------------------------------------

{\1% % this is the name of the PDF file for the background
% the plain option removes the header from the title page, noframenumbering removes the numbering of this frame only
\begin{frame}[plain,noframenumbering] 
  \titlepage % call the title page information from above
\end{frame}
}

\begin{frame}{Summary}{}
  \tableofcontents
\end{frame}

%=======================================================
\section{Introduction}
%=======================================================
\begin{frame}{Introduction}{Overview}
  \begin{figure}[ht]
    \centering
    \includegraphics[width=0.85\textwidth, keepaspectratio=true]{images/introduction.jpg}
  \end{figure}
\end{frame}

\begin{frame}{Introduction}{Overview}
  \begin{figure}[overview]
    \includegraphics[width=0.7\textwidth]{images/paradigms.png}
  \end{figure}
\end{frame}

\begin{frame}{Introduction}{Overview}
  \begin{figure}[overview]
    \includegraphics[width=0.7\textwidth]{images/paradigmAndLanguages.png}
  \end{figure}
\end{frame}

\begin{frame}{Introduction}{Overview}
  \begin{figure}[overview]
    \includegraphics[width=0.7\textwidth]{images/paradigmAndLanguagesAndKuniri.png}
  \end{figure}
\end{frame}

\begin{frame}{Introduction}{Overview}
  \begin{figure}[overview]
    \includegraphics[width=0.7\textwidth]{images/overview.png}
  \end{figure}
\end{frame}

%-------------------------------------------------------
\begin{frame}{Introduction}{About our logo}
  \begin{figure}[overview]
    \includegraphics[width=0.68\textwidth]{images/2.jpg}
  \end{figure}
\end{frame}

%-------------------------------------------------------
\begin{frame}{Introduction}{Curiosity}

  \begin{columns}
  \begin{column}{0.4\textwidth}
    \begin{figure}[fga]
      \includegraphics[width=0.8\textwidth]{images/fga.png}
    \end{figure}
  \end{column}

  \begin{column}{0.4\textwidth}
    \begin{figure}[paulo]
      \includegraphics[width=1.2\textwidth]{images/prmm.jpg}
    \end{figure}
  \end{column}
  \end{columns}

\end{frame}


%-------------------------------------------------------
\begin{frame}{Introduction}{About our organization}
  \begin{figure}[overview]
    \includegraphics[width=0.7\textwidth]{images/github.png}
  \end{figure}
\end{frame}

\begin{frame}{Introduction}{About our organization}
  \begin{figure}[overview]
    \includegraphics[width=0.5\textwidth]{images/codeclimate.png}
  \end{figure}
\end{frame}

\begin{frame}{Introduction}{About our organization}
  \begin{figure}[overview]
    \includegraphics[width=0.8\textwidth]{images/coveralls.png}
  \end{figure}
\end{frame}

\begin{frame}{Introduction}{About our organization}
  \begin{figure}[overview]
    \includegraphics[width=0.9\textwidth]{images/travisci.png}
  \end{figure}
\end{frame}

\begin{frame}{Introduction}{About our organization}
  \begin{figure}[overview]
    \includegraphics[width=0.8\textwidth]{images/rubygems.png}
  \end{figure}
\end{frame}

%=======================================================
\section{Kuniri Architecture}
%=======================================================
\begin{frame}{Architecture}{Overview}
  \begin{figure}[overview]
    \includegraphics[width=0.7\textwidth]{images/scribe_cattermole.jpg}
  \end{figure}
\end{frame}

\begin{frame}[fragile]{Architecture}{Overview - ruby}
  Ruby Code
  \begin{columns}
    \begin{column}{0.6\textwidth}
      \small
\begin{lstlisting}
class MasterClass
  @attrTest
  def initialize
    ...
  end
  def validateInput
    ...
  end
end
\end{lstlisting}
    \end{column}

    \begin{column}{0.4\textwidth}
      Class name: \textcolor{red}{MasterClass}\\ \pause
      Attributes: \textcolor{red}{attrTest}\\ \pause
      Constructor: \textcolor{red}{initialize}\\ \pause
      Methods: \textcolor{red}{validateInput}\\
    \end{column}

  \end{columns}
\end{frame}

\begin{frame}[fragile]{Architecture}{Overview - Java}
  \begin{columns}
    \begin{column}{0.6\textwidth}
      \small
\begin{lstlisting}
class XptoClass{
  private int value;
  XptoClass(){
    ...
  }
  public void validateXpto(){
    ...
  }
}
\end{lstlisting}
    \end{column}

    \begin{column}{0.4\textwidth}
      Class name: \textcolor{red}{XptoClass}\\ \pause
      Attributes: \textcolor{red}{value}\\ \pause
      Constructor: \textcolor{red}{XptoClass}\\ \pause
      Methods: \textcolor{red}{validateXpto}\\
    \end{column}

  \end{columns}
\end{frame}

%-------------------------------------------------------
\begin{frame}{Architecture}{All}
  \begin{figure}[all]
    \includegraphics[width=0.97\textwidth]{images/OverviewClasses.png}
  \end{figure}
\end{frame}

\begin{frame}{Architecture}{Container Data}
  \begin{figure}[containerdata]
    \includegraphics[width=0.93\textwidth]{images/ContainerData.png}
  \end{figure}
\end{frame}

\begin{frame}{Architecture}{Container Data}
  \begin{figure}[classAndFileElement]
    \includegraphics[width=0.9\textwidth]{images/classAndFileElementData.png}
  \end{figure}
\end{frame}

\begin{frame}{Architecture}{Container Data}
  \begin{figure}[overview]
    \includegraphics[width=0.9\textwidth]{images/functionBehaviour.png}
  \end{figure}
\end{frame}


%-------------------------------------------------------
\begin{frame}{Architecture}{Abstract Container}
  \begin{figure}[overview]
    \includegraphics[width=1\textwidth]{images/abstractContainer.png}
  \end{figure}
\end{frame}

\begin{frame}{Architecture}{Template - GoF pattern}
  \begin{figure}[stategof]
    \includegraphics[width=0.9\textwidth]{images/template.png}
  \end{figure}
\end{frame}

\begin{frame}{Architecture}{Abstract Container}
  \begin{figure}[overview]
    \includegraphics[width=0.9\textwidth]{images/abstractContainerClasses.png}
  \end{figure}
\end{frame}

\begin{frame}{Architecture}{Abstract Container}
  \begin{figure}[overview]
    \includegraphics[width=0.9\textwidth]{images/abstractContainerConcrete.png}
  \end{figure}
\end{frame}

\begin{frame}[fragile]{Simple Code}{Demonstration}
Code from:
\verb|lib/kuniri/language/ruby/class_ruby.rb|
\small
\begin{lstlisting}
...
class ClassRuby < Languages::Class
  ...
  def get_class(pLine)
    result = detect_class(pLine)
    return nil unless result

    classCaptured = Languages::ClassData.new
    ...
    return classCaptured
  end
  ...
  def detect_class(pLine)
    regexExpression = /^\s*class\s+(.*)/
    ...
  end
...
\end{lstlisting}
\end{frame}

%-------------------------------------------------------
\begin{frame}{Architecture}{State Machine}
  \begin{figure}[All]
    \includegraphics[width=0.8\textwidth]{images/fsm.png}
  \end{figure}
\end{frame}

\begin{frame}{Architecture}{State Machine - GoF pattern}
  \begin{figure}[stategof]
    \includegraphics[width=0.8\textwidth]{images/state.png}
  \end{figure}
\end{frame}

\begin{frame}{Architecture}{State Machine}
  \begin{figure}[All]
    \includegraphics[width=0.9\textwidth]{images/fsmPack.png}
  \end{figure}
\end{frame}

\begin{frame}{Architecture}{State Machine}
  \begin{figure}[Look at FSM]
    \includegraphics[width=0.9\textwidth]{images/idleToClassToMethod.png}
  \end{figure}
\end{frame}

\begin{frame}{Architecture}{State Machine}
  \begin{figure}[Look at class structure]
    \includegraphics[width=0.9\textwidth]{images/classFSM.png}
  \end{figure}
\end{frame}

\begin{frame}[fragile]{Simple Code}{Demonstration}
Code from:
\verb|lib/kuniri/state_machine/OO_structured_fsm/class_state.rb|
\small
\begin{lstlisting}[language=Ruby, caption=Class State]
...
class ClassState < OOStructuredState
  @language
  def initialize (pLanguage)
    @language = pLanguage
  end
  def handle_line(pLine)
    if @language.line_inspect(AGGREGATION_ID, pLine)
      aggregation_capture
  ...
  end
  def method_capture
    @language.set_state(@language.methodState)
  end
...
\end{lstlisting}
\end{frame}

%-------------------------------------------------------

%=======================================================
\section{Law of Demeter inside Kuniri}
%=======================================================
\begin{frame}{Law of Demeter}{Overview}
\begin{block}{Law Of Demeter}
A method of an object should invoke only the methods of the following kinds of
objects:
  \begin{enumerate}
    \item itself
    \item its parameters
    \item any objects it creates/instantiates
    \item its direct component objects
  \end{enumerate}
\end{block}
\end{frame}

\begin{frame}{Architecture}{Law of Demeter}
  \begin{figure}[Look at class structure]
    \includegraphics[width=0.8\textwidth]{images/demeter_train.jpg}
  \end{figure}
\end{frame}

\begin{frame}{Architecture}{Law of Demeter}
  \begin{figure}[Look at class structure]
    \includegraphics[width=0.8\textwidth]{images/demeter_1.png}
  \end{figure}
\end{frame}

\begin{frame}{Architecture}{Law of Demeter}
  \begin{figure}[Look at class structure]
    \includegraphics[width=0.8\textwidth]{images/demeter_2.png}
  \end{figure}
\end{frame}

\begin{frame}{Architecture}{Law of Demeter}
  \begin{figure}[Look at class structure]
    \includegraphics[width=0.8\textwidth]{images/demeter_3.png}
  \end{figure}
\end{frame}


%-------------------------------------------------------
\begin{frame}[fragile]{Law of Demeter}{Real code}
Code from:
\verb|lib/kuniri/state_machine/OO_structured_fsm/class_state.rb|
\small
\begin{lstlisting}
...
def execute(pElementFile, pLine)
  attributeElement = @language.attributeHandler.get_attribute(pLine)
  if attributeElement
    lastIndex = pElementFile.classes.length - 1
    attributeElement.each do |attribute|
      attribute.comments = @language.string_comment_to_transfer
    end
    @language.string_comment_to_transfer = ''
    pElementFile.classes[lastIndex].add_attribute(attributeElement)
  end
...
\end{lstlisting}
\end{frame}

\begin{frame}[fragile]{Law of Demeter}{Apply LoD}
Code from:
\verb|lib/kuniri/language/container_data/structured_and_oo/|
\small
\begin{lstlisting}
...
class FileElementData < Languages::BasicData
...
 def add_attribute_to_last_class(pAttributeElement)
  classes.last.add_attribute(pAttributeElement)
 end
...
end
\end{lstlisting}
\end{frame}

\begin{frame}[fragile]{Law of Demeter}{Real code}
Code from:
\verb|lib/kuniri/state_machine/OO_structured_fsm/class_state.rb|
\small
\begin{lstlisting}
...
def execute(pElementFile, pLine)
  attributeElement = @language.attributeHandler.get_attribute(pLine)
  if attributeElement
    lastIndex = pElementFile.classes.length - 1
    attributeElement.each do |attribute|
      attribute.comments = @language.string_comment_to_transfer
    end
    @language.string_comment_to_transfer = ''
    pElementFile.add_attribute_to_last_class(attributeElement)
  end
...
\end{lstlisting}
\end{frame}

\begin{frame}[fragile]{Law of Demeter}{Next part}
Code from:
\verb|lib/kuniri/state_machine/OO_structured_fsm/class_state.rb|
\small
\begin{lstlisting}
...
def execute(pElementFile, pLine)
  attributeElement = @language.attributeHandler.get_attribute(pLine)
  if attributeElement
...
\end{lstlisting}
\end{frame}

\begin{frame}[fragile]{Law of Demeter}{Next part}
Code from:
\verb|lib/kuniri/language/language.rb|
\small
\begin{lstlisting}
...
def line_inspect(pTarget, pLine)
  eval("#{pTarget}Handler.get_#{pTarget}(pLine)")
end
...
\end{lstlisting}
\end{frame}

\begin{frame}[fragile]{Law of Demeter}{Next part}
Code from:
\verb|lib/kuniri/state_machine/OO_structured_fsm/class_state.rb|
\small
\begin{lstlisting}
...
def execute(pElementFile, pLine)
  attributeElement = @language.line_inspect('attribute', pLine)
...
\end{lstlisting}
\end{frame}

\begin{frame}[fragile]{Law of Demeter}{Problem...}
  \begin{figure}[overview]
    \includegraphics[width=0.8\textwidth]{images/problem.jpg}
  \end{figure}
\end{frame}

\begin{frame}[fragile]{Law of Demeter}{Next part}
\small
\begin{lstlisting}
time rake

[Coveralls] Set up the SimpleCov formatter.
[Coveralls] Using SimpleCov's default settings.
..................................................

682 examples, 0 failures

Randomized with seed 6659

real 0m7.114s
user 0m6.850s
sys 0m0.227s
\end{lstlisting}
\end{frame}

\begin{frame}[fragile]{Law of Demeter}{Change it}
\small
Code from:
\verb|lib/kuniri/language/language.rb|
\begin{lstlisting}
...
def line_inspect(pTarget, pLine)
  case pTarget
    when StateMachine::METHOD_ID
      return @methodHandler.get_method(pLine)
    when StateMachine::CONSTRUCTOR_ID
      return @constructorHandler.get_constructor(pLine)
...
\end{lstlisting}
\end{frame}

\begin{frame}[fragile]{Law of Demeter}{Change it again}
\small
\begin{lstlisting}
time rake

[Coveralls] Set up the SimpleCov formatter.
[Coveralls] Using SimpleCov's default settings.
..................................................

682 examples, 0 failures

Randomized with seed 36092

real 0m5.139s
user 0m4.867s
sys 0m0.230s
\end{lstlisting}
\end{frame}

%-------------------------------------------------------
\begin{frame}{Law of Demeter}{Singleton and Law of Demeter}
  \begin{columns}[T]
      \begin{column}{.5\textwidth}
    \begin{figure}[overview]
      \includegraphics[width=1\textwidth]{images/singleton.png}
    \end{figure}
      \end{column}
       \hfill
      VS
      \begin{column}{.5\textwidth}
    \begin{figure}[train]
      \includegraphics[width=1\textwidth]{images/demeter_train.jpg}
    \end{figure}
      \end{column}
    \end{columns}
\end{frame}

\begin{frame}[fragile]{Law of Demeter}{Fluent interface and Law of Demeter}
  \begin{block}{Fluent interface}
    The biggest difference in building fluent interfaces is the change
      of using setters. Instead of constructing setters without any return type,
      those setters (or other mutations) will return an object that is most
      likely to be needed next in your code.
  \end{block}
Exemplo
\small
\begin{lstlisting}
product.setWeight(100)
        .setHeight(30)
        .setLength(60)
        .setPrice(10)
...
def setWeight(int weight)
  @weight = weight;
  return self
end
\end{lstlisting}

\end{frame}

\begin{frame}{Law of Demeter}{Fluent interface and Law of Demeter}
  \begin{itemize}
    \item Rigibility: modularity, maintainability.
    \item Fragility: flexibility, reusability.
    \item Immobility: reusability, understandability.
  \end{itemize}
\end{frame}

\begin{frame}{Law of Demeter}{Fluent interface and LoD - Rigidity}
  \begin{columns}[T]
      \begin{column}{.5\textwidth}
    Fluent interface
    \begin{itemize}
      \item Not enforces information hiding. The flow of the method is predetermined.
    \end{itemize}
      \end{column}
       \hfill
      \begin{column}{.5\textwidth}
    Law of Demeter
        \begin{itemize}
      \item Enforces information localization and information hiding.
      \item Emphasis on modularity.
    \end{itemize}
      \end{column}
    \end{columns}
\end{frame}

\begin{frame}{Law of Demeter}{Fluent interface and LoD - Fagility}
  \begin{columns}[T]
      \begin{column}{.5\textwidth}
    Fluent interface
    \begin{itemize}
      \item If you change a return type a method you can break another part.
    \end{itemize}
      \end{column}
       \hfill
      \begin{column}{.5\textwidth}
    Law of Demeter
        \begin{itemize}
      \item Methods not depend on each other.
    \end{itemize}
      \end{column}
    \end{columns}
\end{frame}

\begin{frame}{Law of Demeter}{Fluent interface and LoD - Immobility}
  \begin{columns}[T]
      \begin{column}{.5\textwidth}
    Fluent interface
    \begin{itemize}
      \item The API is primarily designed to be readable and to flow.
    \end{itemize}
      \end{column}
       \hfill
      \begin{column}{.5\textwidth}
    Law of Demeter
        \begin{itemize}
      \item Not simple as Fluent interface (readable)
    \end{itemize}
      \end{column}
    \end{columns}
\end{frame}


%=======================================================
\section{Running kuniri and output}
%=======================================================
\begin{frame}{Running}{Example}
  \begin{figure}[overview]
    \includegraphics[width=0.6\textwidth]{images/terminal.png}
  \end{figure}
\end{frame}

%=======================================================
\section{Conclusions}
%=======================================================
\begin{frame}{Conclusion}{}
  \begin{itemize}
    \item A lot of work;
    \item Future works (improvements and new gems:
      \begin{enumerate}
        \item Improve state machine and ruby parser;
        \item Add new languages under kuniri;
        \item Shell tool;
        \item UML tool;
        \item Documentation tool;
        \item Simple metrics;
      \end{enumerate}
  \end{itemize}
\end{frame}

{\1
\begin{frame}[plain,noframenumbering]
  \finalpage{Thanks!}
\end{frame}}

\end{document}
